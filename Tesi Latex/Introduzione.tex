\chapter{Introduzione}
\label{Introduzione}

\noindent La tesi si prefigge la realizzazione di una applicazione capace di rilevare le informazioni del mondo reale per elaborarle, così da modellare una realtà virtuale.\\

\noindent Si tratta quindi di un'applicazione che permetta all’utente, dotato di visore per la realtà virtuale, di muoversi all’interno di un mondo tridimensionale. La possibilità di movimento è data da una cyclette, che dà all'utilizzatore la sensazione di muoversi lungo un percorso su una bicicletta. La cyclette è fornita di particolari sensori che ne determinano velocità di pedalata e rotazione del manubrio. Lo scopo della tesi è quello di far comunicare questi sensori con un computer in modo da pilotare la bicicletta virtuale.\\

\noindent In futuro si cercherà di utilizzare una bicicletta vera, applicata su rulli che diano un ritorno di forza per implementare la sensazione di affaticamento, a seconda della pendenza del percorso virtuale che viene rilevata dal simulatore.
\newpage
\section{Struttura della tesi}
\noindent La tesi è strutturata nel modo seguente.\\
Nella sezione \textit{\nameref{VR}} si illustrerà lo stato dell'arte nell'ambito della realtà virtuale. Lo studio di questo ambito è stato reso necessario per la progettazione.\\
\noindent Nelle sezioni \textit{\nameref{hardware}} e \textit{\nameref{software}} si mostrerà il progetto nella sua interezza con la descrizione di tutti i moduli.\\
\noindent Nella sezione \textit{\nameref{sperimentazione}} si descriveranno i processi della sperimentazione, tra cui problemi incontrati ed eventuali correzioni ad essi. Si descriveranno inoltre alcune prove di calibrazione dei valori relativi alla fisica del mondo virtuale.\\
\noindent Nella sezione \textit{\nameref{conclusioni}} si riassumono gli scopi, le valutazioni di questi e le prospettive future.







